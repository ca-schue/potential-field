\chapter{Ziel der Implementierung}

Gegebene Aufgabenstellung:

Es soll ein Planungssystem zur Mikronavigation implementiert werden. Die
Planung erfolgt auf Basis eines statisch gegebenen, einfachen Occupancy Gids
mit Hilfe der Potenzialfeldmethode. Der Roboter besitzt eine rechteckige Form
variabler Größer. Der gefundene Plan wird visualisiert. Planungsstatistiken
werden geführt. Die Leistungsfähigkeit des Planers wird anhand verschiedener
Planungsszenarien bewertet.

Umgesetzt durch:
- Python Jupyter Notebook
- Einzige externe Bibliothek zur Durchführung Mathematischer Operationen: numpy
> Fokus der Implementierung
	- Robotermodell: 
		- Rechteckiger Roboter mit Variabler Größe (width, length)
		- Rotation des Roboters um "Ankerpunkt" im und gegen den Uhrzeigersinn in 90° Schritten (0°, 90°, 180°, 270°) 
			> "Ankerpunkt" = Ecke "links oben" bei 0° Rotation
	- Konfigurationsraum:
		- Roboter befindet sich in statischem Occupancy Grid: 2D Array variabler Größe: Wenn sich An Koordinate (X,Y) ein Hindernis befindet => False, sonst True
		- Um Roboterrotationen zu berücksichten: Transformation des 2D Occupancy Grid in einen 3D Konfigurationsraum:
		- Ebene in Z-Richtung entspr. Roboterrotation => 4 Ebenen
		- In jeder Ebene wird Occupancy Grid um die Dimensionen des Roboters in der jeweiligen Rotation erweitert
	- Potenzialfelder:
		- Roboter wird Startpunkt/-rotation und Zielpunkt/-rotation im Occupancy-Grid gegeben
		- Berechnung von Potenzialfeldern in Konfigurationsraum 1) Mit Attractive (anziehend zu Zielpunkt) / Repulsive Potenzialen
	- Roboternavigation:
		- Berechnung der Potenzialgradienten
		- Gradientenabstiegsverfahren
