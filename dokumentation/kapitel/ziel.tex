\chapter{Ziel der Implementierung}

Die vorliegende Arbeit dokumentiert die praktische Umsetzung eines Planungssystems zur Roboternavigation im Rahmen der Vorlesung ``Intelligente Robotik`` im Wintersemester 2023/24.

Die Implementierung erfolgte in Python, wobei ein Jupyter Notebook als zentraler Einstiegspunkt dient. 
Ein rechteckiger Roboter variabler Größe navigiert in einem statisch vorgegebenen Occupancy Grid mit beliebig konfigurierbaren Hindernissen von einem Start- zum Zielpunkt.
Die kollisionsfreie Roboternavigation wird durch die Transformation der Roboterbewegung in einen dreidimensionalen Konfigurationsraum nach einem Ansatz von Yunfeng und Chirikjian ermöglicht \cite{wang.2000}.
Die Routenplanung von Start- zu Zielpunkt basiert auf Potenzialfeldern, deren Berechnung sowohl mit anziehenden und abstoßenden Potenzialen als auch dem Wavefront-Algorithmus implementiert wurde. 
Die Roboternavigation erfolgt durch das Gradientenabstiegsverfahren in den Kraftfeldern der Potenzialfelder. 
Diverse grafische Darstellungen visualisieren die Berechnungen und Roboternavigation.

Nachfolgendes Kapitel dient als Kurzanleitung zur Ausführung des Programms. Die weiteren Kapitel befassen sich mit dem theoretischen Hintergrund der Implementierung.

%Nachfolgende Arbeit dient der Dokumentation der Praktischen Arbeit für die Vorlesung "Intelligente Robotik" im Wintersemster 2023/24.
%- Planungssystem zur Roboternavigation 
%- Umgesetzt in Python. Zentraler Einstiegspunkt: Jupyter Notebook
%- Rechteckiger Roboter mit Variabler Größe (width, length)
%- Translation und Rotation in einem statisch gegebenem Occupancy Grid mit beliebig konfigurierbaren Hindernissen für verschiedene Planungsszenarien.
%- Kollisionsfreie Roboternavigation durch Transformation der Roboterbewegung in dreidimensionalen Konfiguraionsraum gemäß dem Ansatz von XYZ (TODO!).
%- Roboternavigation von Start- zu Zielpunkt basierend auf Potenzialfeldern: In Implementierung sowohl Berechnung mit anziehendem und abstoßendem Potenzialen realisiert als auch über den Wavefront-Algorithmus 
%- Roboternavigation durch Gradientenabstiegsverfahren in Kraftfeldern der Potenzialfelder.
%- Visualisierung der Berechnungen und der Roboternavigation.
%
%Aufbau Arbeit:
%- Nachfolgendes Kapitel = Kurzanleitung zur Ausführung
%- Alle darauf folgenden restlichen Kapitel befassen sich mit dem theoretischen Hintergrund der Implementierung