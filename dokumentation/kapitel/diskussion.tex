\chapter{Diskussion}

\section{Vergleich der Potenzialfeldmethoden}

\subsection*{Anziehendes \& Abstoßendes Potenzial}

In der Literatur bekanntes Problem von anziehendem und abstoßendem Potenzial: 


- Für konkave Hindernisse konvergiert die Potenzialfunktion zu einem lokalem Minimum.

"However, the workspace modeled by APF suffers from local minima, especially when mapping convex-shaped obstacles. Path generation by gradient search over an APF containing local minima will result in the robot getting stuck at some intermediate state before reaching the target location." \cite{maqbool.2021}

	- Beispiel: Literatur "C"- und "U"-shaped Hindernisse \cite{yujiang.2017}


- Oszillationen zwischen Hindernisengstellen:
	Wird deutlich bei Szenarien wenn:
	- Dimensionen des Occupancy Grid klein sind: Da Grenzen als bei Berechnung des abstoßenden Potenzials Hindernisse interpretiert werden: Oszillationen, wenn Roboter zwischen Hindernis und Grenze
	- Roboter gefangen zwischen zwei Gradienten, die aufeinander zeigen

*** TODO: Abbildung von kleinem Occu Grid vs Großes Occu Grid bei gleicher Robotergröße ***


Im Gegnsatz dazu: Wavefront Algorithmus: 
	- Konvergiert nie zu lokalem Minima
	- Potenziale nur in physikalisch erreichbaren Regionen definiert

*** TODO: Abbildungen mit unterschiedlichen Wavefront Szenarien ***


\section{Überdeckungen des Robotermodells mit Hindernissen}

Beobachtung bei den getesteten Szenarien der Implementierung: 
- Schlechte Interpolation der rotierten Robotermaske für kleine Roboterdimensionen mit kleinen Rotationsschritten in einem kleinen Occupancy Grids
 *** TODO: Beispiel 10° ***

In Literatur (Author Wavefront) deshalb empfohlen: mit kleinen Rotationsschritten die Roboterdimensionen und das Occupancy Grid zu skalieren
=> Erhöht die Auflösung der Robotermaske, verringert Artefakte der Interpolation

*** TODO Beispiel Skalierung ***