\chapter{Berechnung der Potentialfelder}

- Anforderung: Die Planung erfolgt [...] mit Hilfe der Potenzialfeldmethode.

Def. Potenzialfeldmethode = jeder Koordinate des diskretisierten Konfigurationsraums wird ein physikalisches Potential zugewiesen:

Vereinfacht: je Höher die "potenzielle Energie" einer Koordinate, desto weiter ist der Punkt auf dem aktuellen Pfad vom Ziel entfernt.

Berechnung dieses Potenzialfelds für den Konfigurationsraum ist Vorbedingung zur Roboternavigation im nächsten Kapitel.

Zur Berechung des Potenzialfelds wurden in dieser Implementierung zwei unterschiedliche Ansätze der Literator verfolgt.

\section{Attractive/Repulsive Potentiale}

Idee: Gesamtpotenzial an einer Koordinate = Kombination aus 
- Anziehendem Potenzial: Berücksichtigt ausschließlich die Entfernung zum Ziel (TODO: Formel)

*** TODO: Abbildung mit Plot ***

- Abstoßendes Potenzial: Hier wird ausschließlich der Einfluss umliegender Hindernisse berücksichtig. (TODO: Unterschiedliche Formeln in Literatur, hier:)

*** TODO: Abbildung mit Plot ***

Kombination der Potenziale unterschiedlich in Literatur. Hier gemäß ... (TODO) Addition beider Potenziale in einem Punkt

*** TODO: Abbildung mit Plot ***

Problem: Lokale Minima


\section{Wavefront Potentiale}

XYZ stellt sogenannte "Wavefront Potenziale" als Alternative zu anziehenden und abstoßenden Potenzialen vor.

Berechnung über sog. "Wavefront" Algorithmus 
- Initialisierung:
	> jeder Punkt mit Potenzial 0 
	> Hindernisse haben nicht definiertes Potenzial (hier np.nan)
	> Warteschlange = [(Zielpunkt, 2)]	

- Wiederhole bis Warteschlange leer:
	> (Aktueller Punkt, aktuelles Potenzial) = Vorderstes Element in Warteschlange
	> Setze aktuelles Potenzial für aktuellen Punkt
	> finde aktuell erreichbare Nachbarn: (x-1, x+1, ... (z-1) mod 4) + kein Hindernis + nicht außerhalb der grenzen des Occupancy-Grids
	> Nachbarn als Tupel in Warteschlange setzen (Nachbar, Potenzial+1)

- Dieser Breitensuchen-ähnliche Algorithmus breitet immer größer werdende Potenziale "wellenartig" im erreichbaren Raum aus
- Punkte, die hinter Hindernissen liegen erhalten somit höhere Potenziale als direkt vom Ziel erreichbare Punkte

*** TODO: Abbildung mit ein paar Iterationen ***



