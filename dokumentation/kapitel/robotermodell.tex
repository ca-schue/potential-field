\chapter{Roboterbewegung im Occupancy Grid}

Die physikalische Umgebung, in der sich der Roboter bewegen kann, wird diskretisiert durch das \textit{Occupancy Grid}.
Dieser binäre, zweidimensionale Raum entspricht der Umgebung aus "Vogelperspektive". 

Das numpy Boolean Koordinatensystem \texttt{occupancy\_grid} mit der Länge \texttt{occupancy\_grid\_length} und Breite \texttt{occupancy\_grid\_width} hat den Ursprung links oben. Bedingt durch Numpy werden Koordinaten mit \texttt{occupancy\_grid[Y][X]} referenziert. Die Umgebung ist im Occupancy Grid binär, wodruch jede Koordinate ($X$,$Y$) durch ein Hindernis belegt ist (\texttt{occupancy\_grid[Y][X] == False}) oder frei von Hindernissen ist (\texttt{occupancy\_grid[Y][X] == True}).

*** TODO: Abbildung 2D True/False Array und geplottetes Occupancy Grid ***

Gemäß der gestellten Anforderungen kann die Dimension des Roboters durch die Variablen \texttt{robot\_width} und \texttt{robot\_length} definiert werden. Pro Verarbeitungseinheit kann sich der Roboter entweder durch eine Translation oder Rotation im Occupancy Grid bewegen:
\begin{itemize}
\item \textbf{Translation} nach links (\texttt{x-1}), rechts (\texttt{x+1}), oben (\texttt{y-1}) und unten (\texttt{y+1})
\item \textbf{Rotation} um einen \textit{Ankerpunkt}. Bei einer Rotation von $0°$ liegt dieser Referenzpunkt in der linken oberen Ecke des Roboters.
\end{itemize}

** TODO: Abbildung Robotermodell + Translation/Rotation ***
