\documentclass[
  11pt,					% Schriftgröße
  paper=a4,
  DIV=13,				% Seitenlayout (Satzspiegel)
  parskip=half,			% Abstand zwischen Absätzen
  %twoside,				% Doppelseitig
  openright,			% neues Kapitel rechts
%  cleardoublepage,
  bibtotoc,				% Literaturverzeichis in Inhaltsverzeichnis
  headsepline,			% Kopfzeilentrennlinie
  headings,	
%  draft,				% Korrekturfassung
  ]{scrreprt}		% scrartcl	

% Eingabecodierung
\usepackage[utf8]{inputenc}

% Schriftcodierung
\usepackage[T1]{fontenc}

% Sprachraum
\usepackage[ngerman]{babel}

% Blindtext
\usepackage{blindtext}
 
% Schrifteinstellungen
\usepackage{lmodern} 		% Vektorschrift
\renewcommand{\familydefault}{\sfdefault} % Serifenlose Schrift
\usepackage{sansmath}  	% Mathe-Schrift ohne Serifen
\sansmath 							% aktiviert serifenlose Matheschrift
\usepackage{microtype}	% harmonische Typenverteilung

% Literatur einbinden
\usepackage{csquotes}	% Steuerung der Anführungszeichen
\usepackage[
  backend=biber,			% Sortier-Compiler
  style=numeric-comp,	% Zitationsstil
  block=ragged,
  ]{biblatex}

\addbibresource{ref/ref_liste.bib}

% Mathemodus
\usepackage{amsmath,amssymb}

% Trennung
\hyphenation{Crash-zo-ne}

% Bilder einbinden
\usepackage{graphicx}
\graphicspath{{bilder/}}
\usepackage{svg}

% Kopf- und Fußzeile
\usepackage[
	headsepline,	% Kopfzeilen-Sepparationslinie
	automark,		% Lebende Kolumnentitel
	]
	{scrlayer-scrpage}
\pagestyle{scrheadings}		
\ofoot*{\pagemark}			
\ohead{\headmark}


% Titelseite



\titlehead{
		\hfil
		  \includegraphics[width=0.3\textwidth]{thi_logo}
		  \hfil
  }


\title{\vspace{5ex} Roboternavigation durch Potentialfelder: Attractive/Repulsive und Wavefront Potentiale}

\subtitle{ \vspace{6ex} \Large Intelligente Robotik WS2023/24\\Praktische Arbeit \vspace{5ex}}

\author{Carl Schünemann (Mat.Nr. $00107827$)}

\date{\today}
  

\begin{document}
  
  % Titelseite anzeigen
  \maketitle
  
  \pagenumbering{Roman}

  
  % Inhaltsverzeichnis
  \tableofcontents
  
  \cleardoubleoddpage
  \pagenumbering{arabic}

  
  % Kapitel einbinden
  \chapter{Ziel der Implementierung}

Gegebene Aufgabenstellung:

Es soll ein Planungssystem zur Mikronavigation implementiert werden. Die
Planung erfolgt auf Basis eines statisch gegebenen, einfachen Occupancy Gids
mit Hilfe der Potenzialfeldmethode. Der Roboter besitzt eine rechteckige Form
variabler Größer. Der gefundene Plan wird visualisiert. Planungsstatistiken
werden geführt. Die Leistungsfähigkeit des Planers wird anhand verschiedener
Planungsszenarien bewertet.

Umgesetzt durch:
- Python Jupyter Notebook
- Einzige externe Bibliothek zur Durchführung Mathematischer Operationen: numpy
> Fokus der Implementierung
	- Robotermodell: 
		- Rechteckiger Roboter mit Variabler Größe (width, length)
		- Rotation des Roboters um "Ankerpunkt" im und gegen den Uhrzeigersinn in 90° Schritten (0°, 90°, 180°, 270°) 
			> "Ankerpunkt" = Ecke "links oben" bei 0° Rotation
	- Konfigurationsraum:
		- Roboter befindet sich in statischem Occupancy Grid: 2D Array variabler Größe: Wenn sich An Koordinate (X,Y) ein Hindernis befindet => False, sonst True
		- Um Roboterrotationen zu berücksichten: Transformation des 2D Occupancy Grid in einen 3D Konfigurationsraum:
		- Ebene in Z-Richtung entspr. Roboterrotation => 4 Ebenen
		- In jeder Ebene wird Occupancy Grid um die Dimensionen des Roboters in der jeweiligen Rotation erweitert
	- Potenzialfelder:
		- Roboter wird Startpunkt/-rotation und Zielpunkt/-rotation im Occupancy-Grid gegeben
		- Berechnung von Potenzialfeldern in Konfigurationsraum 1) Mit Attractive (anziehend zu Zielpunkt) / Repulsive Potenzialen
	- Roboternavigation:
		- Berechnung der Potenzialgradienten
		- Gradientenabstiegsverfahren


  % Kapitel einbinden
  \chapter{Ausführung der Implementierung}

Die Roboternavigation mit Potenzialfeldern wurde in Python als \textit{Jupyter Notebook} implementiert.
Das Programm ist nach Vorbereitung der Ausführungsumgebung in sechs sequenziellen Schritten ausführbar.

\section{Vorbereitung der Ausführungsumgebung}

Das Programm ist entweder als lokales Jupyter Notebook ausführbar oder ohne benötigte lokale Installationen über \textit{Google Colab} in der Cloud ausführbar.

\subsection*{Lokal}
Um das Programm lokal auszuführen, muss das \href{https://github.com/ca-schue/potential-field.git}{GitHub Repository} geklont werden.
Das Jupyter Notebook \texttt{robot-navigation-potential-fields-local.ipynb} ist der zentrale Einstiegspunkt des Programms.
Die Implementierung wurde mit \textit{Visual Studio Code}, kurz \textit{VSCode}, als Ausführungsumgebung des Notebooks getestet.
Dazu wird Python ab Version 3.8 vorausgesetzt (getestet mit Version 3.8.18).
Isolierte \textit{Python Environments} können über \textit{Anacoda Navigator} installiert werden.
Zur Ausführung eines Jupyter Notebooks wird der \textit{ipykernel} benötigt. Dieses Paket kann entweder nach Öffnen des Notebooks in VSCode über das Pop-up installiert werden oder über die Kommandozeile von Anaconda: \texttt{conda install -n <environment\_name> ipykernel}.

\subsection*{Google Colab}
Alternativ kann das Notebook ohne lokale Installationen in \href{https://colab.research.google.com/gist/ca-schue/73cff6faf02b6d75d84573625fd89bea/robot-navigation-with-potential-fields.ipynb}{Google Colab} ausgeführt werden. Hierfür wird einzig ein Google Account benötigt.


\section{Ausführung des Jupyter Notebooks}

Das Jupyter Notebook unterteilt das Programm in sechs sequenziell auszuführende Schritte. Ein Ausführungsschritt entspricht einer Überschrift, die eine auszuführende \textit{Zelle} gruppiert. Per Klick auf das Pfeilsymbol links neben der Überschrift wird die Zelle ausgeklappt. Per Klick auf den Knopf neben der Zelle oder über die Tastenkombination \texttt{Strg + Enter} wird der Python Code ausgeführt:
\begin{figure}[H]
	\centering
	\footnotesize
	\centerline{\resizebox{1\linewidth}{!}{\includegraphics{bilder/cell execution.png}}}
	\caption{Nach Aufklappen der Überschrift eines Ausführungsschritts (links) kann der Python Code der Zelle ausgeführt werden (rechts).}
\end{figure}

\section*{0) Abhängigkeiten installieren}
Bevor das Python Programm ausgeführt werden kann, müssen alle notwendigen Bibliotheken installiert werden. Im Falle der lokalen Ausführung werden hier mit \textit{pip install -r requirements.txt} die benötigten Pakete installiert: \textit{Numpy} und \textit{Scipy} unterstützen für mathematische Berechnungen während \textit{matplotlib} zur Visualisierung dient. Nach Installation der Pakete müssen VSCode und Anaconda Navigator komplett neu gestartet werden.

Beim Notebook für Google Colab wird zusätzlich das Repository geklont. Ein Neustart wird nicht benötigt.

\section*{1) Szenario auswählen}
Dieser Ausführungsschritt gruppiert als Ausnahme mehrere Zellen. Eine Zelle entspricht hier einem vorkonfiguriertem \textit{Szenario}. Dort werden alle Parameter des Programms eingestellt, beispielsweise die Dimensionen des Roboters, dessen Start- und Zielposition und die Position der Hindernisse.
Die Bedeutung der Parameter sind als Kommentare dokumentiert, sodass neue Szenarien hinzugefügt werden können. Zu beachten ist, dass nur die Parameter der zuletzt ausgeführten Zelle übernommen werden.
\begin{figure}[H]
	\centering
	\footnotesize
	\centerline{\resizebox{1\linewidth}{!}{\includegraphics{bilder/parameters.png}}}
	\caption{Über die Parameter wird die gesamte Programmausführung eines Szenarios konfiguriert.}
\end{figure}

\section*{2) Berechnungen durchführen}
Dieser Ausführungsschritt führt die in den nachfolgenden Kapiteln ausführlich beschriebenen mathematischen Berechnungen durch. Diese werden für das Gradientenabstiegsverfahren benötigt. Änderungen am Python Code dieser Zelle sind nicht notwendig.

\section*{3) Plots ausgeben (optional)}
Als einziger Ausführungsschritt, ist die Berechnung und Visualisierung der grafischen Darstellungen optional. Da dieser Schritt sehr rechenaufwändig sein kann, verhindert der im Ausführungsschritts 1) gesetzte Parameter \texttt{optional\_plots=False} bei bestimmten vorkonfigurierten Szenarien die Ausführung dieser Zelle.

\section*{4) Gradientenabstieg initialisieren}
Bedingt durch die \texttt{matplotlib} Bibliothek muss bei animierten grafische Darstellungen die Initialisierung von der Ausführungslogik auf mindestens zwei Zellen aufgeteilt werden. Diese Zelle initialisiert somit das Gradientenabstiegsverfahren und dessen animierte Visualisierung. Wenn im Ausführungsschritt 1) der Parameter \texttt{gradient\_decent\_plots=True} gesetzt wird, werden zusätzliche Plots berechnet. Analog zum Parameter \texttt{optional\_plots=False} des vorherigen Ausführungsschritts wird dies aufgrund des hohen Ressourcenverbrauchs für gewissen Szenarien nicht empfohlen.

\section*{5) Gradientenabstieg starten}
Beim Ausführen dieser Zelle wird das Gradientenabstiegsverfahren gestartet. Im Plot des vorherigen Ausführungsschritts wird die Roboternavigation animiert.
Um ein weiteres Szenario auszuführen, werden die Ausführungsschritte ab 1) wiederholt.




  % Kapitel einbinden
  \chapter{Roboterbewegung im Occupancy Grid}

Die physikalische Umgebung, in der sich der Roboter bewegen kann, wird diskretisiert durch das \textit{Occupancy Grid}.
Dieser binäre, zweidimensionale Raum entspricht der Umgebung aus "Vogelperspektive". 

Das numpy Boolean Koordinatensystem \texttt{occupancy\_grid} mit der Länge \texttt{occupancy\_grid\_length} und Breite \texttt{occupancy\_grid\_width} hat den Ursprung links oben. Bedingt durch Numpy werden Koordinaten mit \texttt{occupancy\_grid[Y][X]} referenziert. Die Umgebung ist im Occupancy Grid binär, wodruch jede Koordinate ($X$,$Y$) durch ein Hindernis belegt ist (\texttt{occupancy\_grid[Y][X] == False}) oder frei von Hindernissen ist (\texttt{occupancy\_grid[Y][X] == True}).

*** TODO: Abbildung 2D True/False Array und geplottetes Occupancy Grid ***

Gemäß der gestellten Anforderungen kann die Dimension des Roboters durch die Variablen \texttt{robot\_width} und \texttt{robot\_length} definiert werden. Pro Verarbeitungseinheit kann sich der Roboter entweder durch eine Translation oder Rotation im Occupancy Grid bewegen:
\begin{itemize}
\item \textbf{Translation} nach links (\texttt{x-1}), rechts (\texttt{x+1}), oben (\texttt{y-1}) und unten (\texttt{y+1})
\item \textbf{Rotation} um einen \textit{Ankerpunkt}. Bei einer Rotation von $0°$ liegt dieser Referenzpunkt in der linken oberen Ecke des Roboters.
\end{itemize}

** TODO: Abbildung Robotermodell + Translation/Rotation ***

  
  % Kapitel einbinden
  \chapter{Konfigurationsraum}

% https://cs.stanford.edu/people/eroberts/courses/soco/projects/1998-99/robotics/definitions.html

Bei einer Roboterlänge und -breite größer als eins können Punkte im Occupancy Grid nicht erreicht werden, ohne mit den Roboterdimensionen Hindernisse oder Grenzen zu überdecken. 
Aus diesem Grund wird die Bewegung des Roboters im Occupancy Grid in eine Punktbewegung im sogenannten \textit{Konfigurationsraum} transformiert.
In diesem Raum entsprechen die Roboterdimensionen einem Punkt (Roboterlänge = Roboterbreite = $1$), was die Überdeckung von Hindernissen und Grenzen verhindert.

Zur Berechnung des Konfigurationsraums wird jedes Hindernis im Occupancy Grid um die Roboterdimensionen erweitert.
Aufgrund der variablen Roboterdimension ändern sich die überdeckten Koordinaten im Occupancy Grid je nach aktueller Rotation.
Ansätze in der Literatur schlagen deshalb vor, unabhängig von der aktuellen Rotation des Roboters, jedes Hindernis in X- und Y-Richtung um einen Kreis mit maximaler Roboterdimension als Radius zu vergrößern.

XYZ (TODO) schlägt einen restriktiveren Ansatz vor, um möglichst wenig Koordinaten um das Hindernis im Konfigurationsraum zu erweitern.
Dazu wird die Roboterrotation in diskrete Schritte unterteilt. Die Variable \texttt{rotation\_step} gibt als Teiler von 360° an, um wie viel Grad sich der Roboter pro Bewegungsschritt drehen kann.
Somit sind $\texttt{rotations} = (360 \text{\textdegree} \div \texttt{rotation\_step})$ unterschiedliche Ausrichtungen des Roboters um den Ankerpunkt möglich.

\begin{figure}[h!]
	\centering
	\footnotesize
	\centerline{\resizebox{0.7\linewidth}{!}{\input{bilder/rotation-steps_latex.pdf_tex}}}
	\caption{Bei $\texttt{rotation\_step}=45$° ergeben sich $(360 \text{\textdegree} \div 45) = 8$ Rotationszustände.}
\end{figure}

Für das Robotermodell wird eine Maske als 2D-Array (\texttt{robot\_length}$*$\texttt{robot\_width}) erstellt und für jeden Rotationsschritt mit \texttt{sklearn.rotate()}\footnote{Mögliche Artefakte und Interpolation der Matrixrotation müssen manuell korrigiert werden.} gedreht.
Für jede der $(360 \text{\textdegree} \div \texttt{rotation\_step})$ möglichen Rotationen wird ein \textit{erweitertes Occupancy Grids} generiert. Dazu werden Hindernisse sowie Grenzen des ursprünglichen Occupancy Grids um die rotierte Maske erweitert.
Beispielsweise Rotation um 45° ...

\begin{figure}[h!]
	\centering
	\footnotesize
	\centerline{\resizebox{0.9\linewidth}{!}{\input{bilder/robot-mask_latex.pdf_tex}}}
	\caption{Erzeugung eines erweiterten Occupancy Grid für eine Rotation um $45$°}
\end{figure}


Die erweiterten Occupancy Grids werden zum Konfigurationsraum \texttt{configuration-space[rotation][y][x]} mit den Dimensionen $\texttt{rotations} * \texttt{occupancy\_grid\_length} * \texttt{occupancy\_grid\_width}$ zusammengefasst. Für die Dimension \texttt{[rotation]} gilt:
\begin{itemize}
\item $\text{[}(\texttt{rotation} + 1) \mod \texttt{rotations]}$ $\widehat{=}$ Rotation um \texttt{rotation\_step} im Uhrzeigersinn
\item $\text{[}(\texttt{rotation} - 1) \mod \texttt{rotations]}$ $\widehat{=}$ Rotation um \texttt{rotation\_step} gegen den Uhrzeigersinn
\end{itemize}

\begin{figure}[h!]
	\centering
	\footnotesize
	\centerline{\resizebox{1\linewidth}{!}{\includegraphics{bilder/configuration-space.png}}}
	\caption{Die erweiterten Occupancy Grids bilden den Konfigurationsraum}
\end{figure}

Dieser 3D Konfigurationsraum ermöglicht die kollisionsfreie Translation und Rotation bei minimalem Verbrauch freier Koordinaten.


  \chapter{Berechnung der Potenzialfelder}

Gemäß der Aufgabenstellung erfolgt die Pfadplanung der Roboternavigation zu einem Zielpunkt mit der \textit{Potenzialfeldmethode}:

Der Konfigurationsraum entspricht einem skalaren \textit{Potenzialfeld}, wobei jede Koordinate eine potenzielle Energie $U(\texttt{x}, \texttt{y}, \texttt{rotation})$ besitzt, ausgedrückt durch eine reelle Zahl.
Je Höher die potenzielle Energie einer Koordinate, desto weiter ist der Punkt auf dem aktuellen Pfad vom Ziel entfernt.
\cite{yujiang.2017}
Die Berechnung des Potenzialfelds $\texttt{potential[rotation][y][x]} = U(\texttt{x}, \texttt{y}, \texttt{rotation})$ erfolgt mit \textit{Potenzialfunktionen}.

\section{Anziehende \& Abstoßende Potenziale}

Khatib schlug 1986 vor, das Potenzialfeld in Analogie zu Gravitationsfeldern zu berechnen. Der zu erreichende Zielpunkt wirkt mit einem \textit{anziehenden} (engl. \textit{attractive}) Potenzial auf eine Koordinate (\texttt{y}, \texttt{x}) \cite{khatib.1985}. Das Potenzial ist in allen Rotationsebenen gleich:
\vspace*{0.2cm}
\begin{equation*}
U_{Attr, Ziel}(\texttt{x}, \texttt{y}) = \sqrt{(\texttt{y} - \texttt{y}_{Ziel})^2 + (\texttt{x} - \texttt{x}_{Ziel})^2}
\end{equation*}

\vspace*{-0.1cm}
Bei den getesteten Implementierungsszenarien wurden bessere Ergebnisse festgestellt, wenn das anziehenden Potenzial erst normiert und anschließend gewichtet wird. Mit $\texttt{attraction\_weight} = 1$ hat das anziehende Potenzial einen Wertebereich von $[0;1]$. Mit  $\texttt{attraction\_weight} < 1$ wird die obere Grenze des Wertebereichs verringert, mit $\texttt{attraction\_weight} > 1$ vergrößert.
\vspace*{0.2cm}
\begin{equation*}
U_{Attr, Ziel}(\texttt{x}, \texttt{y})_{norm} = \frac{U_{Attr, Ziel}(\texttt{x}, \texttt{y})}{\max \{ U_{Attr, Ziel}\}} * \texttt{attraction\_weight} 
\end{equation*}

\vspace*{-0.1cm}
Die Hindernisse im Konfigurationsraum wirken auf jede Koordinate (\texttt{rotation}, \texttt{y}, \texttt{x}) mit einem \textit{abstoßendem} (engl. \textit{repulsive}) Potenzial. Dazu zählen auch die Hindernisse aus der benachbarten Rotationsebene $((\texttt{rotation} + 1) \mod \texttt{rotations})$ sowie $((\texttt{rotation} - 1) \mod \texttt{rotations})$. Mit $\texttt{repulsion\_weight}=0$ hat das abstoßende Potenzial einen Wertebereich von $[0;1]$. Werte $\texttt{repulsion\_weight} > 0 $ verringern die obere Grenze gegen $0$.
\vspace*{0.2cm}
\begin{equation*}
\hspace*{-0.04\linewidth}
\resizebox{1.1\linewidth}{!}{
  $ U_{Repul, Hindernis}(\texttt{x}, \texttt{y}, \texttt{rotation}) = \frac{1}{\texttt{repulsion\_weight} + \sqrt{(\texttt{y} - \texttt{y}_{Hindernis})^2 + (\texttt{x} - \texttt{x}_{Hindernis})^2 + (\texttt{rotation} - \texttt{rotation}_{Hindernis})^2}}
$}
\hspace*{-0.06\linewidth}
\end{equation*}

\vspace*{-0.1cm}
Yujiang und Huilin definieren das abstoßende Potenzial an der Koordinate (\texttt{rotation}, \texttt{y}, \texttt{x}) als kleinsten Abstand zu allen Hindernissen \cite{yujiang.2017}.
\vspace*{0.2cm}
\begin{equation*}
U_{Repul}(\texttt{x}, \texttt{y}, \texttt{rotation}) = \min_{\forall \,\,Hindernis \,\,\in \texttt{ occupancy\_grid}} \{ U_{Repul, Hindernis}(\texttt{x}, \texttt{y}, \texttt{rotation}) \}
\end{equation*}

\vspace*{-0.1cm}
Die gesamte potenzielle Energie einer Koordinate entspricht der Kombination beider Potenziale.
In der Literatur werden dazu unterschiedliche Ansätze vorgeschlagen. Beispielsweise wählt XYZ das Maximum beider Potenziale. In dieser Implementierung wurden anziehendes und abstoßendes Potenzial gemäß Khalib addiert \cite{khatib.1985}.
\vspace*{0.2cm}
\begin{equation*}
U(\texttt{x}, \texttt{y}, \texttt{rotation}) = U_{Attr, Ziel}(\texttt{x}, \texttt{y})_{norm} + U_{Repul}(\texttt{x}, \texttt{y}, \texttt{rotation})
\end{equation*}

\begin{figure}[h!]
	\centering
	\footnotesize
	\centerline{\resizebox{1\linewidth}{!}{\includegraphics{bilder/total-potential-computation.png}}}
	\caption{Die Berechnung des Gestamtpotenzial für die $0$°-Rotationsebene des Konfigurationsraums mit $\texttt{attraction\_weight}=5$ und $\texttt{repulsion\_weight}=0$.}
\end{figure}



\section{Wavefront Potenziale}

Choset stellt die Anwendung der Breitensuche ausgehend vom Zielpunkt als Potenzialfunktion vor. Beim sogenannten \textit{Wavefront-Algorithmus} entsprechen die Koordinaten des Konfigurationsraums den Knoten der Breitensuche, wobei jeder besuchte Knoten das monton steigende Potenzial der jeweiligen Breitensuchenebene erhält \cite{choset.2007}:

\begin{algorithm}
\caption{Wavefront-Algorithmus}
\begin{algorithmic}[1]
    \State \textbf{Initialisierung:}
    \State \hspace{\algorithmicindent} Jeder Punkt $U(\texttt{x}, \texttt{y}, \texttt{rotation}) = 0$
    \State \hspace{\algorithmicindent} Warteschlange $Q := \{((\texttt{x}_{Ziel}, \texttt{y}_{Ziel}, \texttt{rotation}_{Ziel}), 2)\}$
	\vspace*{0.3cm}
    \While{$Q \neq \emptyset$}
        \State $((\texttt{x}, \texttt{y}, \texttt{rotation}), \texttt{potential}) \gets Q$
        \State $U(\texttt{x}, \texttt{y}, \texttt{rotation}) = \texttt{potential}$
        \State Nachbarn $N := \{(\texttt{x-1}, \texttt{y}, \texttt{rotation}), (\texttt{x+1}, \texttt{y}, \texttt{rotation}), ... (\texttt{x}, \texttt{y}, \texttt{(rotation - 1) \% rotations})\}$
        \For{$(\texttt{x}_{Nachbar}, \texttt{y}_{Nachbar}, \texttt{rotation}_{Nachbar}) \gets N$}         
            \If{$0 \leq \texttt{x}_{Nachbar} < \texttt{occupancy\_grid\_width}$ \\
                \hspace*{\algorithmicindent}\hspace*{\algorithmicindent} \textbf{and} $0 \leq \texttt{y}_{Nachbar} < \texttt{occupancy\_grid\_height}$ \\
                \hspace*{\algorithmicindent}\hspace*{\algorithmicindent} \textbf{and} $ \texttt{computational\_space}[\texttt{rotation}_{Nachbar}][\texttt{y}_{Nachbar}][\texttt{x}_{Nachbar}] = \texttt{True}$}
                \State $((\texttt{x}_{Nachbar}, \texttt{y}_{Nachbar}, \texttt{rotation}_{Nachbar}), \texttt{potential + 1}) \rightarrow Q$
            \EndIf
        \EndFor
    \EndWhile
\end{algorithmic}
\end{algorithm}

Somit breitet sich ausgehend vom Zielpunkt als Quelle mit jeder Iteration das monoton steigende Potenzial bildlich als "Wellenfront" im Konfigurationsraum aus.
Koordinaten in Hindernisnähe erhalten auf entgegengesetzter Seite der Ausbreitungsrichtung höhere Potenziale.

\begin{figure}[h!]
	\centering
	\footnotesize
	\centerline{\resizebox{1\linewidth}{!}{\includegraphics{bilder/wavefront.png}}}
	\caption{Die Berechnung des Gestamtpotenzial für die $0$°-Rotationsebene des Konfigurationsraums mit $\texttt{attraction\_weight}=5$ und $\texttt{repulsion\_weight}=0$.}
\end{figure}

\vspace*{1cm}

Unabhängig von der gewählten Potenzialfunktion wird das Potenzialfeld im dreidimensionalen Konfigurationsraum für jede Rotationsebene berechnet.
Somit hat das Potenzialfeld die gleichen Dimensionen wie der Konfigurationsraum:

\begin{figure}[h!]
	\centering
	\footnotesize
	\centerline{\resizebox{1\linewidth}{!}{\includegraphics{bilder/potential-stacked-computed.png}}}
	\caption{Das Potenzialfeld wird im dreidimensionalen Konfigurationsraum berechnet.}
\end{figure}




  \chapter{Roboternavigation}


Basierend auf berechneten Potenzialen für jede freie Koordinate im Konfigurationsraum: Roboternavigation durch Gradientenabstiegsverfahren = In jedem Punkt in die Richtung der größten Verringerung des Potenzials gehen.

Fokus der Implementierung: Beheben lokaler Maxima (durch Kräftegleichgewicht)

\section{Berechnung der Gradienten}

Berechnung der Gradienten über np.gradients => Gradienten = Kraftvektoren in x, y und Rotations-Richtung
Zeigen in Richtung der größten Verringerung des Potenzials der benachbarten Koordinaten entlang einer Achse

*** Abbildung mit Kraftvektoren in Konfigurationsraum ***

Problem 1: np.gradients berechnet Gradienten an Grenzen von Array als Differenz der letzten beiden Werte (edge-order = 1)
Lösung: Manuelles clipping der Gradienten: Wenn Gradient in Richtung der Grenze => Wird auf 0 geclippt
Wichtig: durch Clipping können an Grenzen können lokale Minima oder Plateaus geschaffen werden:
	*** TODO: Abbildung ***


\subsection{Gradienten an Hindernissen}

Problem 2: np.gradients Gradienten an Hindernissen (np.nan) np.nan als Gradienten. 
Lösung: Behandlung von Hindernissen als Array-Grenzen: (Funktion "compute-obstacle-gradients")
	- "Grenze rechts" (x+1 == np.nan): Gradient = ... Beispiel: *Abb*
	- "Grenze links" (x-1 == np.nan): Gradient = ... Beispiel: *Abb*
	- "Grenze vorne" (y-1 == np.nan): Gradient = ... Beispiel: *Abb*
	- "Grenze hinten" (y+1 == np.nan): Gradient = ... Beispiel: *Abb*
	- "Grenze oben" (z-1 mod 4 == np.nan): Gradient = ... Beispiel: *Abb*
	- "Grenze unten" (z+1 mod 4 == np.nan): Gradient = ... Beispiel: *Abb*
Wichtig: durch Clipping können an Hindernissen lokale Minima und Plateaus geschaffen werden: Wenn beide anderen Kraftvektoren ebenfalls 0 sind 


\subsection{Behandlung lokaler Maxima}

Problem: lokale Maxima
> In jeder Achse haben die beiden entgegengesetzten Nachbarn gleich großes Potenzial
> Unterschied lokales Minimum und Plateau: Potenzial der Nachbarn ist dabei niedriger als das der aktuellen Position

Lösung:
> Durch Implementierung von "compute-obstacle-gradients" kann lokales Maximum nicht an Grenze oder Hindernis existieren
> Gedachte Grenze zu einem Nachbarn setzen: Berechne Differnz zu einem der Nachbarpotenziale Gradienten

*** Abbildung mit Beispiel ***

\section{Gradientenabstieg}

Lokales Minimum oder Plateau: 0-Kraftvektor in allen Achsen

Startpunkt in 3D Konfigurationsraum

Wiederhole bis Kraftvektor x = Kraftvektor y = Kraftvektor rotation = 0:
	- Diskretisierung der Gradienten in Translationen des 3D Konfigurationsraum: Für betragsmäßig größten (oder gleich große) Gradienten in x, y oder Rotations-Richtung wird neue Position berechnet
	- Aus allen möglichen neuen Positionen wird diejenige gewählt, die noch nicht besucht wurde und deren Gesamtkraft maximal ist-

*** Abbildung mit paar Iterationen bis Ziel gefunden ***


\section{Vergleich der Potenzialfeldmethoden}

Anziehendes/Abstoßendes Potenzial:
	- Nachteil : lokale Minima
	...
	- Nicht-Lineare Gradienten


- Vorteil Wavefront Algorithmus: 
	- Garantierte Konvergenz zum Ziel => Keine Lokalen Minima
	...
	Lineare Gradienten
  
  \include{kapitel/grenzen}

  % Literatur anzeigen
  \printbibliography
  
\end{document}